\begin{resumo}[ABSTRACT]
  \begin{SingleSpacing}
    The presence of increasingly sensitive electrical and electronic loads in the current electrical network, makes it essential to provide quality in the power supply, without interruptions or large variations in voltage values. Therefore, the National Electric Energy Agency (ANEEL) proposes regulatory procedures for distribution networks called Electricity Distribution Procedures in the National Electric System (PRODIST), which enable the operation and maintenance of these systems. In particular, module 8 of PRODIST is dedicated to verifying the Power Quality (PQ), establishing limits and indicators ensure the quality of the energy supply and its safety for the consumer and the concessionaire. The study of PQ is usually carried out using network analyzers associated with software, which is generally not very intuitive, increasing the possibility of human error in the analysis of indicators. In this context, this work proposes the development of a routine in the Python programming language with a graphical interface, which generates a power quality report for a consumer unit, automating the process and reducing the possibility of errors in data interpretation. The software was developed based on PRODIST and tested using real data. The tests carried out proved the functionality of the routine, thus providing a power quality report for a consumer unit.
  \end{SingleSpacing}

  \vspace{\onelineskip}
  \textbf{Key words}: Power quality. Python. Graphical interface. Automation.
\end{resumo}