\chapter{CONCLUSÃO}

% Essa conclusão pode melhorar bastante, tente usar o seguinte roteiro por parágrafo:

% 1- introduzir rapidamente o problema em um parágrafo pequeno, terminando com Nesse contexto, o trabalho propôs o desenvolvimento... (com seu primeiro parágrafo)

Levando em consideração que vários dispositivos elétricos são sensíveis a variação de tensão e corrente, procedimentos são adotados pela ANEEL para caracterizar a qualidade de energia elétrica. Nesse contexto, o trabalho propôs o desenvolvimento de uma rotina em Python que efetue essa caracterização para uma unidade consumidora.

% 2- Descrever como o problema foi resolvido, a rotina criada, a divisão dela em classes, a criação da interface gráfica para facilitar o uso.

Partindo disso, foi elaborado um código em Python, com abordagem modular e composta por classes, tendo uma classe geral para conexão das demais, responsáveis pelo cálculo dos indicadores, com a interface gráfica desenvolvida para facilitar o uso.

% 3 - Explicar que dados reais foram utilizados para o desenvolvimento E testes da rotina, e que foi possível exibir as análises a medida que se usa o software como também gerar gráficos e o relatório de QEE com base no prodist.

Para o teste da rotina foram utilizados dados reais obtidos por meio de um analisador de rede dessa própria instituição, em dois pontos de acoplamento e épocas distintas. Para análise dos dados, a medida que se utiliza os \textit{software}, gráficos podem ser gerados, facilitando a classificação dos níveis de tensão. Ademais, é possível gerar um relatório de QEE com base nos índices e limites estabelecidos no PRODIST.

% 4 - Ultimo parágrafo é de fato a conclusão sobre: Portanto, a rotina de computacional elaborada nesse trabalho foi capaz de facilitar a análise de QEE e as hipóteses estabelecidas como objetivos a serem alcançados. Espera-se que o produto gerado possa auxiliar pesquisas e análises de QEE futuramente.

Portanto, a rotina computacional elaborada nesse trabalho foi capaz de facilitar a análise de QEE e as hipóteses estabelecidas como objetivos foram alcançadas. Espera-se que o produto gerado possa auxiliar pesquisas e análises de QEE futuramente.

% No decorrer desse trabalho foi desenvolvida uma rotina em Python contendo uma interface gráfica, com o intuito de caracterizar a qualidade de energia elétrica de uma unidade consumidora seguindo os procedimentos adotados pela ANEEL para tal.

% Os resultados obtidos mostraram que a rotina desenvolvida conseguiu calcular os indicadores de qualidade de energia e gerar o relatório. Conseguindo mostrar também que a interface gráfica implementada facilitou muito a visualização das analises.

% Esse trabalho conseguiu ser de grande importância para facilitar as analises de qualidade de energia. Sendo assim, a proposta desejada foi alcançada e esse trabalho conseguiu realizar o seu objetivo.