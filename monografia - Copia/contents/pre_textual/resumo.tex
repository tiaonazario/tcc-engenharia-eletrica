\begin{resumo}[RESUMO]
  \begin{SingleSpacing}
    A presença de cargas elétricas e eletrônicos cada vez mais sensíveis dentre as cargas utilizadas atualmente na rede elétrica, torna imprescindível o fornecimento de uma energia de qualidade, sem interrupções ou grandes variações nos valores de tensão. Isto posto, a Agência Nacional de Energia Elétrica estabelece procedimentos regulatórios para redes de distribuição, denominados Procedimentos de Distribuição de Energia Elétrica no Sistema Elétrico Nacional (PRODIST), que viabilizam a operação e manutenção desses sistemas. Em especial, o módulo 8 do PRODIST se dedica a verificação da qualidade de energia elétrica (QEE), estabelecendo limites e indicadores para assegurar o fornecimento de energia com qualidade e segurança para o consumidor e a concessionária. O estudo de QEE se dá usualmente por analisadores de rede associados a \textit{softwares}, que são, em geral, pouco intuitivos, aumentando a possibilidade de erros humanos na análise dos indicadores. Nesse contexto, o presente trabalho aborda o desenvolvimento de uma rotina na linguagem de programação Python, com uma interface gráfica, e gera um relatório de análise da qualidade de energia de uma unidade consumidora, automatizando o processo e reduzindo a possibilidade de erros na interpretação dos dados. O \textit{software} foi desenvolvido com base no PRODIST e testado utilizando dados reais. Os testes elaborados comprovam a funcionalidade da rotina, elaborando um relatório acerca da qualidade de energia da unidade consumidora.
  \end{SingleSpacing}

  \vspace{\onelineskip}
  \textbf{Palavras-chave}: Qualidade de Energia Elétrica. Python. Interface Gráfica. Automatização.
\end{resumo}